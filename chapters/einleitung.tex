\chapter{Einleitung}
\label{cha:Einleitung}

\section{Problemstellung}
\label{sec:Problemstellung}

Die Firma netbase GmbH aus Berlin betreut als Großkunden die Yokogawa Deutschland GmbH. Zu deren Hauptgeschäftsfeldern gehören die Herstellung und der Vertrieb von verteilten Prozessleitsystemen sowie von Druck- und Durchflussmessgeräten. Für die Verwaltung und Konfiguration dieser Geräte wurde von der netbase GmbH eine Individualsoftware mit der Bezeichnung FlowConfigurator entwickelt. Diese Software, ermöglicht es den Vertriebsmitarbeiter der Yokogawa Deutschland GmbH anhand von Kundenspezifikationen den geeigneten Sensor aus dem umfangreichen Produktbestand zu bestimmen und anschließend zu konfigurieren. Dabei findet ein vielschichtiges Filtersystem Anwendung, welches logisch abstrahiert in einer Datenbank organisiert ist. Die Filterdaten umfassen dabei sowohl gerätespezifische Merkmale, als auch Zuordnungen zu Sensoren und Produktkategorien, auf Grundlage derer die Sensorenauswahl eingeschränkt werden kann. Die Filtersteuerelemente in Form von Dropdown-Listen und Schaltern werden auf Grundlage dieser Datenbasis zur Laufzeit in der FlowConfigurator Software generiert und angeordnet. 

Die Anforderungen an die vorhandenen Filter können sich jedoch ändern oder müssen erweitert werden, wenn zum Beispiel neue Produkte eingeführt werden. Dies verursacht erhöhten Pflegeaufwand auf Seiten der netbase GmbH, da die Filter bisher nur direkt in der Datenbank angepasst und erweitert werden können. Durch die Entwicklung einer Verwaltungsoberfläche zur Pflege und Organisation der Filterdaten soll dieser Prozess in Zukunft effizienter und robuster gestaltet werden.

\newpage

\section{Zielsetzung}
\label{sec:Zielsetzung}

Ziel im Rahmen dieser Arbeit ist die prototypische Entwicklung einer Software zur Pflege von Filterdaten des FlowConfigurator. Diese Software soll als Webapplikation technisch getrennt von der FlowConfigurator Software realisiert werden. Die Datenbasis des FlowConfigurator wird dabei in Form einer \ac{MSSQL} Datenbank von der netbase GmbH bereitgestellt. Für die Realisierung des Projekts, gilt es die technischen und theoretischen Grundlagen zu erarbeiten. Ein Schwerpunkt ist die Analyse der vorliegenden Datenstruktur und die darauf aufbauende Funktionsweise des FlowConfigurator hinsichtlich Benutzerführung und Filterlogik. Da die Filterdaten nicht nur aus Merkmalen und Verknüpfungen bestehen, sondern zusätzlich auch Layout- und Typinformationen der zu generierenden Filtersteuerelemente beinhalten, ist es notwendig, neben Dateneingabemasken für Zuordnungen und Eigenschaften, auch Konzepte für die Anpassung der layoutspezifischen Daten zu entwickeln. Insbesondere ist hier die ebenfalls anpassbar zu machende Anordnung und Gruppierung der Filtersteuerelemente zu nennen. Aus diesem Grund soll die zu gestaltende Benutzeroberfläche auf dem \ac{WYSIWYG} Prinzip aufbauen und der des FlowConfigurator in Bezug auf Layout und Funktionsweise ähneln. Dies soll eine eingängliche und effektive Anpassung der Filterdaten, insbesondere der Anordnung der Filtersteuerelemente, ermöglichen. 

Wie bereits in Abschnitt \ref{sec:Problemstellung} angedeutet liegen die Filter als Datenrepräsentation eines komplexen und vielschichtigen Filtersystems vor. Deshalb ist es für die Spezifikation einer Schnittstelle notwendig, Abhängigkeiten auf Datenbankebene zu identifizieren und bedarfsabhängig zusammenzufassen.  Ebenso muss eine Homogenisierung der Daten in Betracht gezogen werden, um eine möglichst leichtgewichtige und einheitliche Schnittstelle zu schaffen. Diese Schnittstelle soll die Grundlage eines Webservice bilden, der vom Client konsumiert wird. Der Webservice soll außerdem vor Fremdzugriffen geschützt werden und deshalb ein rudimentäres Authentifizierungssystem implementieren.

Nach dem Bearbeiten der Filterdaten sollen diese durch einen Datenexport im FlowConfigurator wieder nutzbar gemacht werden. Das übergeordnete Ziel ist dabei sowohl der Erhalt der Datenintegrität als auch die Sicherstellung einer korrekten Funktionsweise der Sensorfilterung und Darstellung der Filtersteuerelemente hinsichtlich eines möglichen angepassten Layouts.

\newpage

\section{Aufbau der Arbeit}

Die Arbeit beginnt mit der Betrachtung von thematischen Grundlagen, die zum Verständnis der Arbeit beitragen. Eine intensive Betrachtung des \acf{REST} Architekturstils im Kontext einer \acf{ROA} bildet den Einstieg in diese Arbeit. Anschließend wird das Wissenschaftsgebiet Software-Ergonomie anhand des Schwerpunktes Grundsätze der Dialoggestaltung beleuchtet und in diesem Zusammenhang eine Übersicht über Evaluations- und Prüftechniken gegeben.

Das dritte Kapitel der Arbeit beinhaltet die Anforderungsanalyse. Hier werden Anwendungsfälle beschrieben, die von dem zu entwickelnden Prototyp abgedeckt werden und Anforderungen definiert. Des Weiteren findet eine Betrachtung und Vergleich von in Frage kommenden Technologien statt.

Gegenstand des vierten Kapitels ist zum einen die Beschreibung der gewählten Software-Architektur, als auch ein Entwurfskonzept der zu gestaltenden Benutzeroberfläche. Weiterhin werden Überlegungen zum Schnittstellenentwurf diskutiert und in diesem Zusammenhang das zugrunde liegende Datenmodell beschrieben.

Im fünften Kapitel wird auf Grundlage der vorher getätigten Analysen und Entwürfe die Implementierung des Prototyps beschrieben. Dabei wird auf die Besonderheiten der Umsetzung eingegangen und das Zusammenspiel der eingesetzten Technologien betrachtet. 

Das sechste Kapitel beinhaltet den Test des Prototyps. Die angewandte Testmethode wird mit Bezug auf den Versuchsaufbau beschrieben und das zu erwartende Ergebnis benannt. Eine Testauswertung stellt abschließend den Bezug zum tatsächlichen Testergebnis her.

Den Abschluss bildet eine Bewertung des entwickelten Prototyps, sowie eine Zusammenfassung zur Erweiterbarkeit und bestehenden Problemen der aktuellen Implementierung.