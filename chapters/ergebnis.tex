\chapter{Ergebnis}
\label{cha:Ergebnis}
Als Abschluss der vorliegenden Arbeit wird in diesem Kapitel auf die erreichten Ziele eingegangen, sowie ein Ausblick für mögliche Verbesserungen gegeben.

Das Ziel der Arbeit war, eine Webanwendung zu entwickeln mit der es möglich ist komplexe Filterdaten, im Kontext der von der netbase GmbH entwickelten FlowConfigurator Software, zu verwalten. Zu diesem Zweck wurde die bestehende Softwarelösung und Datenbasis analysiert und daraus Anforderungen an die Webanwendung abgeleitet. Darauf aufbauend erfolgte ein Systementwurf, in dem mit Hilfe von Diagrammen die Anwendung modelliert und dargestellt wurde. Ein Schwerpunkt dieses Entwurfs war die Gestaltung der Benutzeroberfläche auf Grundlage von WYSIWYG-Prinzipien. Anschließend wurde der Systementwurf implementiert und dessen Umsetzung anhand von Quellcodebeispielen erläutert. Abschließend wurde anhand einer vorgestellten Testmethode die Funktionsfähigkeit des entwickelten Prototypen untersucht.

\section{Bewertung}

Anhand von Vorbetrachtungen und Analysen der Funktionsweise der FlowConfigurator Software auf Grundlage einer bestehenden Datenbasis, konnte gezeigt werden, wie sich bereits bestehende Prozesse bei der Verarbeitung komplexer Datenstrukturen als Ausgangspunkt für die Entwicklung einer Datenpflegeanwendung nutzen lassen. Die aus der Analyse der Datenstruktur und der Betrachtung der Benutzeroberfläche der FlowConfigurator Software gewonnenen Erkenntnisse konnten abstrahiert als Teile eines Systementwurfs verwendet werden. Der entworfene Prototyp ist in seiner Anwendung sehr flexibel gestaltet worden. Die starke Trennung der Systemkomponenten und die Realisierung eines vielseitig ansprechbaren REST konformen Webservice ermöglichen es, die Anwendung mit geringem Aufwand weiter zu entwickeln und zu modifizieren. Die Benutzeroberfläche konnte nahe am Vorbild der FlowConfigurator Software konzipiert werden, ohne dabei auf Vorteile einer webgestützen Anwendung verzichten zu müssen. Die dabei verwendeten WYSIWYG-Konzepte für die Anpassung von Filtersteuerelementen fügen sich dank modernen und interaktiven Drag-and-Drop Mechanismen nahtlos in den Bearbeitungsvorgang der Filterdaten ein. Die direkte Manipulation von Filtersteuerelementen sind dabei eine sinnvolle Ergänzung zu der ebenfalls implementierten und auf Dateneingabemasken gestützten Bearbeitung der Filterdaten. Die in Abschnitt \ref{sec:Funktionale Anforderungen} erarbeiteten funktionalen Anforderungen konnten vollständig implementiert werden. Ausnahme ist hier die Anforderung \emph{Models hinzufügen}, die aus zeitlichen Gründen nicht erfolgen konnte, von dem Systementwurf aber abgedeckt wird. Außerdem bestehen Probleme bei der datengetriebenen Generierung der Filtersteuerelemente in der Webanwendung. Diese Probleme begründen sich im angewandten Grid-System zur Platzierung der Filtersteuerelemente, da die Datenbasis auf ein Fließlayout ausgerichtet ist. Die an die Anwendung gestellten Tests konnten aus diesem Grund nur mit Einschränkung bestanden werden. Der Systementwurf und die Implementierung demonstrieren dennoch, wie die Pflege komplexer Datenstrukturen unter Anwendung von interaktiven Bearbeitungskonzepten auf Grundlage des WYSIWYG-Prinzips umgesetzt werden kann.

\section{Ausblick}

Im Folgenden werden Themen genannt, auf Grundlage derer sich der entwickelte Prototyp erweitern beziehungsweise verbessern lässt.

Aufgrund komplexer Datenabfragen und nicht optimierter Algorithmen beim Zugriff auf die entwickelte Schnittstelle kommt es in der Webanwendung gelegentlich zu erhöhten Ladezeiten. Durch eine Identifizierung der Schwachstellen kann die Performance des Webservice noch erheblich optimiert werden, um eine reibungslose Bearbeitung der Filterdaten in der Webanwendung zu fördern. Weiterhin weist der entwickelte Algorithmus, für die Konvertierung der layoutspezifischen Filterdaten von einem Fließlayout hin zu einem Grid-Layout Schwachstellen auf. Der Algorithmus basiert bereits auf einem generischen Ansatz, der aufgrund der Komplexität der zugrunde liegenden Daten, aber noch nicht mit allen auftretenden Randbedingungen umgehen kann. Außerdem wurde serverseitig nur eine prototypische Implementierung für die Validierung der zu persistierenden Daten realisiert. Um schwere Ausnahmefehler auf Datenbankebene zu verhindern und ein besseres Fehlerverhalten zu gewährleisten ist es notwendig, Fehler frühzeitig abzufangen und aussagekräftige Fehlermeldungen zu generieren. Insgesamt ist damit festzustellen, dass das Potential der entwickelten Anwendung noch nicht ausgeschöpft ist und in seiner prototypischen Realisierung viel Raum für Erweiterungsmöglichkeiten vorhanden ist.