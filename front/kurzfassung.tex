\chapter*{Kurzfassung}

In der vorliegenden Arbeit wird die Entwicklung einer Verwaltungsplattform für die Pflege komplexer Datenstrukturen beschrieben. Dabei handelt es sich konkret um Filterdaten einer bereits bestehenden Software für die Bestimmung und Konfiguration von Druck- und Durchflussmessgeräten, die von der Yokogawa Deutschland GmbH hergestellt werden. Die entstehende Verwaltungsplattform wird durch eine Webanwendung realisiert und die dafür benötigte Kommunikation mit einem Server mittels eines Webservice umgesetzt.

Ein Schwerpunkt der Arbeit ist die Analyse bestehender Datenstrukturen und die darauf aufbauende Konzeption einer REST-konformen Schnittstelle. Die zu gestaltende Benutzeroberfläche realisiert dabei Konzepte der datengetriebenen Generierung von Steuerelementen auf Grundlage der gegebenen Datenbasis. Aufgrund der geteilten Datenbasis zwischen der bestehenden Konfigurationssoftware und der zu entwickelten Webanwendung werden außerdem \acf{WYSIWYG} Gestaltungskonzepte angewendet, um eine effiziente und robuste Datenpflege zu realisieren.


